%%%%%%%%%%%%%%%%%%%%%%%%%%%%%%%%%%%%%%%%%%%%%%%%%%%%%%%%%%%%%%%%%%%%%%%%%%%%%%%%
% Template for USENIX papers.
%
% History:
%
% - TEMPLATE for Usenix papers, specifically to meet requirements of
%   USENIX '05. originally a template for producing IEEE-format
%   articles using LaTeX. written by Matthew Ward, CS Department,
%   Worcester Polytechnic Institute. adapted by David Beazley for his
%   excellent SWIG paper in Proceedings, Tcl 96. turned into a
%   smartass generic template by De Clarke, with thanks to both the
%   above pioneers. Use at your own risk. Complaints to /dev/null.
%   Make it two column with no page numbering, default is 10 point.
%
% - Munged by Fred Douglis <douglis@research.att.com> 10/97 to
%   separate the .sty file from the LaTeX source template, so that
%   people can more easily include the .sty file into an existing
%   document. Also changed to more closely follow the style guidelines
%   as represented by the Word sample file.
%
% - Note that since 2010, USENIX does not require endnotes. If you
%   want foot of page notes, don't include the endnotes package in the
%   usepackage command, below.
% - This version uses the latex2e styles, not the very ancient 2.09
%   stuff.
%
% - Updated July 2018: Text block size changed from 6.5" to 7"
%
% - Updated Dec 2018 for ATC'19:
%
%   * Revised text to pass HotCRP's auto-formatting check, with
%     hotcrp.settings.submission_form.body_font_size=10pt, and
%     hotcrp.settings.submission_form.line_height=12pt
%
%   * Switched from \endnote-s to \footnote-s to match Usenix's policy.
%
%   * \section* => \begin{abstract} ... \end{abstract}
%
%   * Make template self-contained in terms of bibtex entires, to allow
%     this file to be compiled. (And changing refs style to 'plain'.)
%
%   * Make template self-contained in terms of figures, to
%     allow this file to be compiled. 
%
%   * Added packages for hyperref, embedding fonts, and improving
%     appearance.
%   
%   * Removed outdated text.
%
%%%%%%%%%%%%%%%%%%%%%%%%%%%%%%%%%%%%%%%%%%%%%%%%%%%%%%%%%%%%%%%%%%%%%%%%%%%%%%%%

\documentclass[letterpaper,twocolumn,10pt]{article}
\usepackage{usenix2019_v3}

% to be able to draw some self-contained figs
\usepackage{tikz}
\usepackage{amsmath}
\usepackage{graphicx,hyperref}
\usepackage{comment}
\graphicspath{{images/}}
\usepackage[utf8]{inputenc}
\usepackage[T1]{fontenc}
\usepackage[french,english]{babel}

\newcommand{\paa}[1]{{\textcolor{red}{[[#1 -- paa]]}}}
\newcommand{\kam}[1]{{\textcolor{blue}{[[#1 -- kam]]}}}
\newcommand{\ash}[1]{{\textcolor{violet}{[[#1 -- ash]]}}}
\vbadness=99999  

\begin{document}
%-------------------------------------------------------------------------------

%don't want date printed
\date{}

% make title bold and 14 pt font (Latex default is non-bold, 16 pt)
\title{\Large \bf Trace forensics}

%for single author (just remove % characters)
\author{
{\rm Your N.\ Here}\\
Your Institution
\and
{\rm Second Name}\\
Second Institution
% copy the following lines to add more authors
% \and
% {\rm Name}\\
%Name Institution
} % end author

\maketitle

%-------------------------------------------------------------------------------
\begin{abstract}
%-------------------------------------------------------------------------------

To respond to an outage or determine the cause of failure, operators and/or developers of the system typically reason both about the presence or absence of events and their interactions in an anomalous execution when compared with the normal behavior of the system. Distributed traces capture causality of events occurring across system components and can therefore be used to troubleshoot outages, since anomalous executions must surely differ from steady state executions - in the events, their interactions, or other attributes such as CPU or memory usage. In order to find \textit{how} the trace from an anomalous execution diverges from steady state, it may be necessary to reason about the similarities and differences across many traces. Current tooling provides features to extract and summarize features for each individual traces, but provides little support for reasoning and analysis across traces. By contrast, analysis tools that allow efficient comparisons among traces use data representations, such as flattening graphs using a depth first walk, that lose valuable \textit{structural} information regarding causal relationships among the recorded events.

In this paper, we explore the construction and use of structural embeddings, borrowed from the NLP literature, to represent and compare traces in a way that preserves their causal context. We present quantitative results demonstrating that the learned embeddings outperform the state of the art for traces drawn from execution paths which correspond to functionality such as checking out an item, viewing the product page and listing item(s) to sell. We also present qualitative results demonstrating the application of our technique to provide hints for troubleshooting outages.



\end{abstract}

%-------------------------------------------------------------------------------
\section{Introduction}
%-------------------------------------------------------------------------------

%Individual services are typically augmented with logging and telemetry but even when taken together, reconstructing end-end request flows in a system is generally infeasible. 

 Tracing is increasingly being used in industry~\cite{Jaeger, Zipkin, 36356} and has been shown to be useful in finding the causes of performance problems~\cite{36356, Fonseca:2007:XPN:1973430.1973450, Reynolds:2006:PDU:1267680.1267689, DBLP:conf/sigmetrics/ThereskaSSWALG06, Sambasivan:2011:DPC:1972457.1972463}, anomaly detection~\cite{Barham:2003:MOM:1251054.1251069, Chen:2004:PFE:1251175.1251198, 37477, Sambasivan:2011:DPC:1972457.1972463} as well as to model workloads, resource usage and timings~\cite{Mann:2011:MPE:2170444.2170464, Chanda:2007:WTP:1272996.1273001, 36356, DBLP:conf/sigmetrics/ThereskaSSWALG06, Barham:2003:MOM:1251054.1251069}. The opentracing standardization effort~\cite{OpenTracing} is another indication that tracing is gaining traction in industry. Furthermore, the proliferation of tracing into components like databases, queues, proxies, cache~\cite{OpenTracingCassandra, OpenTracingKafka, OpenTracingRedis} has led to end to end tracing in distributed systems which enables developers to look at end-to-end properties of a system over multiple executions.

An individual execution is represented in a trace as a directed acyclic graph whose nodes represent discrete labeled events and whose edges represent causal connections among these events.  For example, such a graph might describe how a request to a web service was satisfied via a tree of calls to remote services that provide parts of the 
content of the final rendered page; in some sense, the graph \emph{explains} how the request succeeded, which services were involved, where the time was spent, and so on. Distributed tracing has become a big business.  Many real-world deployments~\kam{cite} and several start-ups focusing narrowly on tracing infrastructure and analysis provide evidence of the perceived value of end-to-end tracing.  

Unfortunately, practitioners often have difficulty extracting that value. Part of the problem is that traces can in general be large and difficult to visualize.  More fundamentally, however, most of the practical problems that practitioners would like to solve with end-to-end tracing do not involve detailed inspection of an individual trace, but rather in \emph{comparing} an individual trace to some \emph{family} of existing traces. 

Consider a site reliability engineer debugging a failure in a distributed system in which a user request fails due to \emph{component misbehavior} resulting in a site outage. An example of such misbehavior is that of a fallback having been misconfigured such that failover no longer works as expected. In this instance, it is insufficiently to merely obtain a trace; rather the SRE would like to know how traces of a failed execution i.e. ones in which the fallback did not take over on failure \emph{differ} from those of successful execution(s) i.e. execution(s) in which fallback behaved correctly. Such a comparison would help determine the \textit{proximal} cause of failure i.e. the most likely cause,  given that there may be several candidate causes.

%An additional constraint is that we need to find traces of successful executions that exercise the same execution path as the failed execution in order to use tracing effectively for debugging. 
%We need to be able to compare graphs in two different ways:
%\begin{itemize}
%\item Comparing traces with one another in order to be able to bin then based on their execution paths.
%\item Differencing of trace of a failed execution from that of successful execution(s) in the same class.
%\end{itemize}

Another use case is \emph{stratified sampling}. Distributed tracing systems like Dapper~\cite{36356} perform uniform sampling at very low rates to keep the cost of storing traces low.  As a consequence, the long tail of distributed executions is not captured. This tail contains traces which are symptomatic of edge cases like slow queries, one-off execution paths that are infrequent but imperative for identifying the bottlenecks and improving the system. One way to bias the sampling so as to collect these rare traces is to test \emph{how different} a given trace is to some corpus of traces from a trailing window.

%\ash{We jumped from a generic failure to long tail distribution. We need to stick to one.
%Please look at Section 2 of So, you want to trace your distributed system?
%Key design insights from years of practical experience? By Sambasavin and pick the use case you want to specifically address.}

%n order to be able to effectively sample traces we need to techniques to compare graphs with one another:
%\begin{itemize}
%\item To identify graphs that are different from each other in order to store traces that appear infrequently.
%\item To identify graphs that are similar to each other and quantify this similarity. 
%\end{itemize}

As a third example, a developer or administrator might, given a new trace of unknown class, wish to \emph{classify} it based on a corpus of labeled examples. All of the use cases involve differential reasoning \emph{across} possibly a large number of graphs. Therefore, in order for tracing to achieve its full potential, we need to couple tracing with graph comparison. 

%As a framework to discuss different graph comparison approaches, we define:
%\begin{itemize}
%\item Trace representation: \newline
%Given a trace associated with a graph, G=(V,E), a trace representation is a map M, such that \newline
%$M\colon G \rightarrow Rep$, where $Rep \in String, R^{d}$
%\item Distance measure to compute distances between the representations
%\end{itemize}

Prior work~\cite{Barham:2003:MOM:1251054.1251069, Sambasivan:2011:DPC:1972457.1972463} use flattening the graph generated by a trace as embedding and string-edit distance as distance measure to compare graphs. More recently, Las-Casas et al.~\cite{Las-Casas:2018:WSE:3267809.3267841} use a node-based array as representation, in which the number of occurrences of a node is an entry in the array and euclidean distance for the distance measure. While their results are promising, a different trace embedding may be necessary based on the characteristics of the graphs or problem under consideration. All the representations used in prior work have been hand-generated using domain knowledge to solve a particular problem.

%Since we primarily perform structural comparisons of the graphs derived from traces, general purpose embeddings that encode structural information can potentially be used for a variety of downstream tasks. Can general purpose trace embeddings automatically from data? 
\emph{Representational learning} primarily emerged to address the problem of feature engineering for graphs before machine learning tasks can be applied. It is attractive for two reasons: It is data-driven and does not depend on feature engineering, unlike prior approaches; nodes or entire graphs are embedded in a low dimensional space $R^{d}$, such that the geometric relationships in the vector space reflect the structure of the original graph. Representational learning techniques have been used in diverse fields such as social network analysis and molecular biology\kam{cite}. We are the first to demonstrate their application to distributed tracing. 
%A recent survey paper by Hamilton et al.~\cite{corr_2017_abs-1709-05584} points out that prior work on feature engineering involved extracting summary information from graphs, graph kernels or carefully engineered features to measure local neighborhood structures. 

In this paper, we explore the use of \emph{representational learning} as a common solution to the uses cases described.  We observe that distributed traces, far from being mere ``bags'' of events, are in some sense \emph{utterances} over a language of events in which structural context carries significant (if difficult to manually extract) meaning, akin to the way that documents are utterances over a language of words. Inspired by advancements in the natural language processing community, we ask whether it is possible and practical to learn contextual embeddings of events akin to Word2Vec, and to construct aggregate representations of entire traces and ultimately collections of similar traces.

We make 3 contributions: \newline
\begin{itemize}
\item We demonstrate the applicability of structure-aware embeddings to trace data 
\item Using clustering as an example, we show that the performance of structure-aware embeddings outperforms the state of the art which loses all  structural information
\item Present qualitative results of the application of our technique to troubleshoot outages
\end{itemize}




























\section{Motivating example}

The crux of the trace forensics problem is: How can we provide hints of proximal causes when an execution fails? 

\begin{figure}[h]
\center{\includegraphics[scale=0.25]{anon_adrbk_fail.pdf}}
\caption{Example graph drawn from trace of a failed interaction. The red lines indicate that the callee has returned an error message to the caller}
\label{Failed_ex}
\end{figure}

To explain the triage process followed by site reliability engineers (SRE) when a failure occurs, we consider the example of a user trying to buy items from an online store. A successful checkout occurs when the user is able to place an order, else the checkout fails. Users being unable to checkout represents an outage of the system. Figure~\ref{Failed_ex} shows the  call-graph of a user interaction involving checkout of items for a cart, which ultimately fails. When users are unable to buy items form the store, it is the job of a site reliability engineer (SRE) to find out why. 
Currently, the SRE sees a number of alerts arising from failures, possibly in RPC calls highlighted as red in the call graph, from the monitoring system.  Subsequently, based on the alerts, the SRE may consult aggregate data over sliding windows of time for services deemed problematic.
The SRE might go about their job like so:
\begin{itemize}
\item Use domain knowledge about the dependencies between services to know which alerts are the result of transitive dependencies and must be ignored. \newline
As an example, the load balancer (LB) in the figure will be alerting. 
An SRE with a good mental model of the system would conclude that since checkout (Chkout) has multiple downstream alerts, the alerts at LB are probably a result of the error being propagated up and try to dig deeper into one of the alerts from the downstream services. The reasoning is that a downstream service generating an alert is more likely to be a candidate for the proximal cause of the problem.  
Once a promising alert has been found, it is now time to look at the logs from the appropriate time to try to determine cause of failure.
\item Determine if the failure of some service(s) is the proximal cause \newline
Sometimes, the outage is caused by failure of mandatory services or a fallback path not being taken. Details of a network connection failure or time out in attempting to connect to downstream services might be buried deep in the logs, taking longer to unearth and fix, resulting in larger user downtimes. 
\end{itemize}

For outages which are not a result of service failures, a user entering invalid payment details perhaps or infrastructure failures, for example, appropriate next steps need to be taken. However, if the SRE determines service failure(s) to be proximal cause, she now has to \emph{obtain and compare} the trace of failed execution with that of one or more successful executions exercising the same request path. However, since we witness a large number of successful executions during steady state, we can learn models from steady state data such that we construct embeddings for graphs derived from the traces. The embeddings serve two purposes:
\begin{enumerate}
\item Embeddings for traces can be classified before they are stored making it easy to find the most appropriate traces to compare to when a failure occurs
\item As the number as well as the size of traces increase, storing the embeddings may be much more compact than the original traces. \kam{Since we haven't tried reversing our embedding to recapture the graphs, there may be some information loss here which we will need to quantify while making this claim}
\end{enumerate}

In the next section, we describe our technique for generation of trace embeddings and the corresponding distance measure used. In the evaluation, we describe how we use the fairly simple operations on the embeddings to address the problem described above and automatically generate hints about the proximal causes for the failure observed. This is made possible by the fact that the embeddings encode information of structural neighborhoods.


%-------------------------------------------------------------------------------
\section{Methodology}
%-------------------------------------------------------------------------------
Graph representation:  Aggregate of its component node (service) representations

Aggregator function: Summation

While there are other aggregator functions that we can use, summation is the simplest and has been used to obtain graph representations in previous work\kam{TODO: Needs citation}. The aggregate representation obtained can also be intuitively thought of as a linear combination of all the nodes present into the graph. 

Service representations : Obtained by using prior work, node2vec\kam{TODO citation}, which employs similar techniques as word2vec but for graphs. Graph nodes, in this case services, are analogous to words, and random walks in the graph generate for us sentences in the doc to train.


%-------------------------------------------------------------------------------
\section{Evaluation}
%-------------------------------------------------------------------------------
For our evaluation, we obtained traces from a large e-commerce company, where the underlying system is a micro-service architecture. The traces encapsulate the request-response interactions of micro-services. We obtained traces from four execution paths in the system that correspond to viewing the produce page, checking out an item in the catalog and two paths which correspond to listing an item to be sold. The execution paths are not mutually exclusive i.e. they share 30-50\% of services and traces from the same path can vary subtly based on the exact request parameters. Furthermore, some pairs of  paths have a higher degree of overlap in the services than others. As mentioned in  section~\ref{methodology}, we compute trace embeddings by aggregating node embeddings. In particular, for every trace, we sum the node embeddings of the nodes present in the trace to obtain the corresponding graph embedding. Since the process of generating node embeddings involves tuning hyper-parameters such as dimension of embeddings, length of random walk and number of walks, we generate node embeddings several (< 10) times. In each instance, we compute the corresponding trace embeddings. To compare a pair of traces using their embeddings, we use Euclidean distance as the distance measure.

%Since clustering of traces based on execution paths depends on graph structure, it serves as the reference task. Trace embeddings that perform best on clustering ate used for other downstream tasks such as classification and fault diagnostics.  
\kam{TODO: How many traces are used to generate node embeddings? We currently use 200 traces to learn embeddings, and the total number of traces is ~4500. We haven't yet verified how many of these traces are duplicates.}
  
\subsection{Clustering}
 We use graphs corresponding to a small fraction of traces are used to generate node embeddings
For a graph G, a naive approach might be to encode the trace as a node array, where the number of occurrences of an event is indicated by an integer n. This baseline embedding R is a vector of dimension D, where D is the the number of unique events. 
Let $E_{i},\; i \in {0..D-1} $ represent the set of unique events. Further, let $Map(E_{i}),\; i \in {0..D-1} $ be the mapping from events to array indices, such that $ Map(E_{i}) \neq Map(E_{j}),\;  i \neq j,\; 0 \leq i,j \leq D-1 $ We now have: \newline
\begin{math}
m_{i} =\; Map(E_{i})\; \forall i \in {0..D-1} \\
R[m_{i}] = n_{i},\;  
\end{math}
where $n_{i}$ is the number of occurrences of the event $ E(m_{i})$\\ 
 Clustering traces by execution paths can be used to determine the number of distinct paths as well as to be able to easily produce examples of successful executions of the system executing a particular path. The computed trace embeddings represent points in $R^{d}$ space, for which we obtain clusters using the K-means approach, with K = 3. 
 
 \kam{\textbf{MUST-TRY}If we swept the value for K, find the K for which the clusters are most stable and then compare predicted values to expected values, this will allow us to say that we can determine the number of distinctive request paths. This is a useful contribution so long as request paths are not mutually exclusive.}
 
 \begin{figure}
\includegraphics[scale=0.2]{tsne_viz.png}
\caption{2-D representation of clusters learned by K-means with graph embeddings as input}
\label{Clusters}
\end{figure}
 
Since we know the labels for the traces a priori, we can compute the Agjusted-Rand\kam{may need to cite} measure, which is a bounded value between [-1,1], where 1 is the perfect score. \kam{Consider computing number of false positives and false negatives instead. Gives us better interpretability} Trace embeddings which perform best return a perfect score. The high dimensional data points are projected to 2-D using t-SNE and distinctive clusters can be seen in figure. Though we are able to learn clusters based on the data accurately, there is a large spread within individual clusters, which translates to poor results for predictive analysis. That is to say, predicting which cluster a new trace belongs to returns inaccurate results for a significant percentage of cases\kam{Need to quantify significant. I think it is more than 50\%, but I don't recall off hand}

\kam{\textbf{MUSTDO}: Draw traces from more than three execution paths. Not only will that add credibility, we also expect to show that the clusters of families of traces that share more services are closer than other clusters. Steps} \newline
\kam{
\begin{enumerate}
\item Pull traces from different request paths from production
\item Since tracing can return too few nodes, determine the minimum number of nodes which should be present in a trace for us to include the trace. Does this need to be a path specific or agnostic?
\item Given a bunch of traces to consider, weed out the duplicates
\item Now randomly shuffle the graphs and learn node representations. Aggregate to obtain trace embeddings. 
\item Cluster the embeddings using K-means. Since we know the labels for each of the traces a priori, we can compute an adjusted-rand measure, which is a bounded value between [-1,1], with 1 being the perfect score. 
\item With NodeCount as graph representation, what is the best value of adjusted rand measure? Number of false positives and false negatives?
\end{enumerate}
} \newline
\kam{Expectation: Graph embeddings performs better than NodeCount. This is the result we saw in the experiment we ran, but that was only one experiment} \newline
\kam{Plot the number of false positives and false negatives for the two approaches as the corpus of traces increases in size. We expect embeddings to perform objectively better than graphKernels and numbers to grow as the size of corpus increases and then level off} \newline

\kam{\textbf{SHOULD DO} \newline Randomization} \newline
\kam{Steps: \newline
\begin{enumerate}
\item Subtractive randomness: Some number of spans may be missing in a number of traces, which we are dubbing subtractive randomness. Is this signifiant enough to impact our baseline performance? Am how do embeddings perform? The expected result here is that the performance of embeddings stays near constant, while that of our baseline deteriorates proportional to the degree of randomness. We only want to include this graph if the deterioration is noticeable.
\item Additive randomness: Add random noise into the traces, such that there are a few nodes in each trace. Is this realistic? 
\item Additive + Subtractive randomness
\end{enumerate}
} 

\kam{\textbf{TODO: Exploratory}\newline XTrace data (old and new)\newline
The embeddings will not learn different clusters unless the DAG of executions are structurally different. So far, the XTrace data does not look significantly different and embeddings perform poorly on these. Work in progress.
}

\kam{\textbf{TODO: Exploratory}\newline Clustering on Uber graphs)\newline
Need to clean up code base and write code for parsing and processing data based on the snippet that Jonathan sent me. Work in progress.
}

\subsection{Classification}
We split our graphs into a training set and test set~\kam{size of training and test set} and are able to learn a logistic regression classifier that is able to predict the class of any test graph with accuracy close to 100\%, as can be seen in Figure~\ref{Classification}. Classification works near-perfectly even when the number of dimensions for node embeddings is as low as 4.

\begin{figure}
\includegraphics[scale=0.5]{test_logistic_2d.png}
\caption{Logistic regression}
\label{Classification}
\end{figure}

\kam{\textbf{MUSTDO} \newline The classification was setup to classify graphs of failed traces and was learning on graphs of successful traces. Apart from the fact that our criteria for failed and successful traces need to be revisited, we might want to set this problem up slightly differently. Set up k-fold cross-validation with k=10. }\newline
\kam{Expected result: Performance of classification keeps getting better as we increase k. Try for values of k = 5,10,15,20,25. If the performance is noticeably different, we can plot a graph for precision and recall of classifier.}

\subsection{Automated Fault diagnostics}
In section 2, we presented the call graph of a trace from a failed execution obtained during an outage. Multiple services were alerting due to downstream failures and that the SRE deems the possible cause of failure to be downstream service(s) not being called. The next step for the SRE is to form a hypothesis and validate it by pulling up logs of one or more affected services. With trace embeddings, we compute the pairwise euclidean distance of the trace embedding from the failed execution with that of every observed successful execution exercising the same request path. In so doing, we avoid computing more conventional distance measures such as graph edit distance, which has been shown to be NP-complete. \kam{cite}

\begin{figure}
\includegraphics[width=0.5\textwidth]{anon_adrbk_repair.pdf}
\caption{Services to be added are in solid green, those to be deleted in solid red. Edges to be added are in dashed green, those to be deleted are in dashed red. Solid green edges represent status needs to be successful instead of failed. TODO: This is an older repair image, needs to be updated. }
\label{RootCauseHint}
\end{figure}

Figure~\ref{RootCauseHint} demonstrates the additions and deletions suggested to the original call graph that would make this interaction successful, in this case, a successful checkout. 

\kam{Investigate: Some completions are worse than others. Why? For most cases, if we did compute graph edit distance, we would get the best results. Results obtained from representations are slightly off. But in 1 out of the 5 cases, the result appears to be really off. Maybe the corpus is too small?} \newline

\kam{TODO: Distinguish between failed and successful executions in different domains. When we obtain traces from production, we may also need to do some work to determine if the traces was successful or not.  This may involve calling out to external data sources.}\newline

\kam{\textbf{SHOULD DO}\newline Failed executions from injected faults. Steps} \newline
\kam{
\begin{itemize}
\item For e2e tests in checkout, inject faults taking down the mandatory services one at a time to obtain call graphs for failed executions for multiple request paths. 
\item Run viewitem test and inject faults in mandatory services for viewitem as well. 
\item Compute trace embeddings for the traces collected.
\item Filter successful traces so that for each trace drawn from a failed execution, it is only compared to successful traces exercising the same request path. 
\item For each failed execution, find the closest successful execution i.e. the one with the minimum euclidean distance. 
\item Compute the graph edit distance of the trace as determined by our technique and compare it to the minimal graph edit distance
\end{itemize}
}

\kam{\textbf{COULD DO} \newline Collect real outage data from looking at eBay's JIRA. Steps} \newline
\kam{
\begin{enumerate}
\item For every outage in the last 90 days, is this an outage we can potentially fix?
\item For every outage we can potentially fix, can we produce the outage using a test and injected faults?
\item Given a trace of an outage, does the repair suggested point to its root cause?
\item Can we quantify how long it takes us to suggest a hint as opposed to amount of time that was spent on the issue?
\end{enumerate}
}
\kam{Results we want to report: \newline
\begin{enumerate}
\item Percentage of total outages we can provide hints for
\item Of the outages we can provide hints for, present overall time taken by our technique to provide hint v/s time spent triaging the outage
\item Ballpark figures of revenue costs associated with the outages that we might have been able to help with
\end{enumerate}
}

\kam{\textbf{COULD DO} \newline As we have described above, we need to compute pairwise distance of a trace from a failed execution to all successful executions exercising the same request path. While we avoid computing graph edit distance, we also don't want to compare with EVERY successful execution exercising the same request path. If we are able to separate graphs of failed executions from successful executions, then we can compare only with a few canonical graphs of successful executions.} \newline

\kam{\textbf{Will not do now} \newline Can we re-construct graphs from embeddings? Information loss? This is more of a research question. How much information do we lose from throwing away traces? What does this mean in terms of recovering the graph from the embedding? } \newline
\kam{\textbf{Will not do now} \newline Using a different aggregator? This is a more long term goal. It is a good  idea to compare with a different aggregator just to see how each performs. But more importantly, if there is too much information loss with summation, we may not be able to re-generate the graph from the embedding. What aggregator will allow us to recover graph from embedding? And what are its applications to graph generation? } \newline

\kam{Sum up the eval}





















%-------------------------------------------------------------------------------
\section{Related Work}
%-------------------------------------------------------------------------------

Post factum analysis of traces is important because they contain useful information like causality of events, resource consumption, errors, performance data and labeled metadata about components involved in the execution.
In large systems the size of the traces makes it impractical for a developer to do any manual analysis. This motivates the need for automated analysis of large corpus of traces to derive useful insights.
There are many goals for analysis for traces, including finding performance bottlenecks, system recovery \& repair, debugging, resource usage, service degradation analysis.
Many of these use cases use some form of classification or clustering. We describe some of the prior work in this area and highlight the drawbacks and explain how our approach overcomes some of the hurdles.

Magpie \cite{Barham:2003:MOM:1251054.1251069} relies on structured events produced logged by ETW (event tracing for windows) in order to avoid the need to have unique identifiers for every request.
Magpie correlates events generated by the application, middleware and operating system using temporal joins to infer causal relationships. 
Magpie uses a comparison based on a simple string-edit-distance metric on flattened execution graphs as a basis for execution clustering.
Even though the technique shows promising results even for classification of requests with differences in internal concurrency structures, the authors acknowledge the need for graph and tree edit distances.

Pinpoint \cite{Chen:2004:PFE:1251175.1251198} collects execution traces as a series of paths through the system. The authors diagnose anomalies by generating a probabilistic context-free grammar (PCFG) from the paths. 
They perform anomaly detection at runtime, whereby new paths are compared to the generated model to determine the probability with which it would be produced from the grammar. 
Anomaly detection worked well in experiments, but the authors note that there are a number of realistic scenarios where it would
not work well: features must be represented in the training set for them to be considered at runtime; changes
such as software upgrade require the model to be retrained; and the learned model represents a superset of observed paths.

Spectroscope \cite{Sambasivan:2011:DPC:1972457.1972463} collects execution traces represented as process invocation trees, and diagnoses performance
changes by comparing sets of before and after traces. Spectroscope assumes that a similar workload was run before and after the performance change, and that the performance change manifests as a change in distribution
over the request structures and/or request timings. To diagnose a change, spectroscope compares the distributions of service completion times for graphs that are topologically identical, and compares structural differences
between executions using string-edit-distance.

Mann et al. \cite{Mann:2011:MPE:2170444.2170464} collect execution traces from datacenter services, and model the latency of a service given the child services invoked.
This work focuses on learning join-point locations by comparing large volumes of Dapper traces.
The execution graphs recorded do not fully capture the causal dependencies internal to a service, so one component of the work is to deduce those causal dependencies from a collection of training examples.
The training examples are then clustered if they have identical execution graphs. 
At runtime, a cluster is selected by comparing its service timings with the cluster centroids, and selecting the nearest-neighbour. 
A prediction for the execution’s overall runtime is then given based on the other executions in the selected cluster.

Mann et al. and Spectroscope do clustering, but only for graphs that are isomorphic.
For graphs with different topology, Spectroscope and Magpie use string-edit distance on canonicalized and linearized graphs as a metric,
for proper clustering in the latter, and for finding similar clusters in the former.
Magpie and Pinpoint cluster similar behavior and suggest outliers as possible indicators of bugs.

Pip \cite{Reynolds2006PipDT} takes a different approach to solving these problems. 
Unlike Magpie, Pip does not rely on any statistical inference and hence obviates the need to collect a large corpus of traces.
Pip treats the systems as \textit{grey-box} and finds structural and performance bugs in distributed systems by comparing actual system behavior to expected system behavior.
Pip asks the developers to specify the expectations of the program using a declarative specification language, while capturing the actual system behavior using instrumentation, annotations or sniffing.
Pip automatically checks actual behavior against expected behavior and helps programmers visualize the result to discover the causes of any unexpected behavior.
However, pip cannot be used for analysis of running programs since it waits for the program to terminate before reconciling traces. 
This makes pip unsuitable for use in online services. 

Las-Casas et al. \cite{Las-Casas:2018:WSE:3267809.3267841} have described initial work on using incremental heirarchical clusterting to solve the problem of representative trace sampling.
The technique uses the PERCH \cite{Kobren:2017:HAE:3097983.3098079} algorithm for incremental heirarchical clustering and can be used for both offline and online sampling.
The technique offers considerable improvements over random sampling for capturing anomalous traces as well as reconstruction of traces from a sample.
For reconstruction of traces they also rely on euclidean edit distances, which we seek to avoid by using representational learning.

\kam{
The other line of related work which I think will be useful to mention is that there has been some work in the systems community around using NLP techniques for systems problems. 
I attended a talk in CIDR where word2vec was utilized for database workload management and have included it in the references. We might want to look around if there is some other work that I've missed.}

%-------------------------------------------------------------------------------
\section{Conclusions and Future Work}
%-------------------------------------------------------------------------------
\begin{itemize}
\item Representations:
    \begin{enumerate}
    \item Use a different aggregator function other than sum:
    This will probably be necessary for synthetic trace generation. Currently, we cannot convert a representation into a trace. 
    \item Assign weights to services based on their importance.
    \item Use tranductive models to get representations for services not seen in training.
    \end{enumerate}
\item Fault diagnosis
    \begin{enumerate}
    \item We are curating our corpus to add more failed graphs. Want to show examples suggesting a successful trace in which a fallback path was taken as a potential hint.
    \item We need to remember some corpus of successful graphs to compare against. The next step would be to learn a few canonical graphs representing successful executions that we can use when providing hints.
    \end{enumerate}
\item Response time mutations
For fault diagnostics, we only considered cases when the unsuccessful execution was not calling out to some expected services. There are failures in which the only differential is response time. For such failure cases, might incorporating latency into representations be a good idea?
\item Other technqiues
    \begin{enumerate}
    \item Graph convolutional networks and autoencoders may be other techniques that may be applicable in this context
    \item Comparisons of these via-a-vis our current technique might be useful to highlight strengths and weaknesses of each.
    \end{enumerate}
\end{itemize}




%-------------------------------------------------------------------------------
%\section*{Acknowledgments}
%-------------------------------------------------------------------------------

%-------------------------------------------------------------------------------
\bibliographystyle{plain}
\bibliography{references}

%%%%%%%%%%%%%%%%%%%%%%%%%%%%%%%%%%%%%%%%%%%%%%%%%%%%%%%%%%%%%%%%%%%%%%%%%%%%%%%%
\end{document}
%%%%%%%%%%%%%%%%%%%%%%%%%%%%%%%%%%%%%%%%%%%%%%%%%%%%%%%%%%%%%%%%%%%%%%%%%%%%%%%%

%%  LocalWords:  endnotes includegraphics fread ptr nobj noindent
%%  LocalWords:  pdflatex acks
