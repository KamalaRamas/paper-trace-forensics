%-------------------------------------------------------------------------------
\section{Introduction}
%-------------------------------------------------------------------------------


End-to-end distributed tracing systems have gained popularity recent years.
By recording and propagating the causal connections between events distributed across components,
these systems in many cases obviate the need ad-hoc approaches for correlating information across logs, to address
problems such as performance profiling, worload monitoring and anomaly detection. (cite)
An individual execution is represented in a trace as a directed acyclic graph
whose nodes represent discrete labeled events and whose edges represent causal connections among these events.  For example, such a graph
might describe how a request to a web service was satisfied via a tree of calls to remote services that provide parts of the content of the final
rendered page; in some sense, the graph \emph{explains} how the request succeeded, which services were involved, where the time was spent, and so on.
Distributed tracing has become a big business.  Many real-world deployments~\cite{} and several start-ups
focusing narrowly on tracing infrastructure and analysis provide evidence of the perceived value of 
end-to-end tracing.  

Unfortunately, practitioners often have difficulty extracting that value.  Part of the problem is that traces can in general be large and difficult
to visualize.  More fundamentally, however, most of the practical problems that practitioners would like to solve with end-to-end tracing
do not involve detailed inspection of an individual trace, but rather in \emph{comparing} an individual trace to some \emph{family} of existing traces.
For example, a developer or administrator might, given a new trace of unknown class, wish to \emph{classify} it based on a corpus of labeled examples.
As another, a site reliability engineer might, in an attempt to discover the root cause of a site outage, want to understand how a given trace obtained
when the site is not operating correctly \emph{differs} from the family of traces obtained when the system is healthy.  
A third example is \emph{stratified sampling}.  Tracing systems control their overheads by collecting traces only from a small sample of executions, meaning
that traces of anomalous executions (which are by assumption unlikely) may be overlooked.  One way to bias the sampling so as to collect these rare traces
is to test \emph{how different} a given trace is to some corpus of traces from a trailing window.
All of these use cases involve
differential reasoning \emph{across} possibly large numbers of traces.

While there has been prior work on techniques for trace comparison based on string edit distance of the flattened graph~\cite{Barham:2003:MOM:1251054.1251069, Sambasivan:2011:DPC:1972457.1972463} or on aggregate measures such as node intersection~\cite{Las-Casas:2018:WSE:3267809.3267841}, we believe that
there is still significant room for improvement.  In particular, these approaches lose contextual information about the \emph{structure} of the 
trace graphs that in our experience is critical for comparison and differentiation.

In this paper, we explore the use of \emph{representational learning} as a common solution to these problems.  We observe that distributed traces,
far from being mere ``bags'' of events, are in some sense \emph{utterances} over a language of events in which structural context carries
significant (if difficult to manually extract) meaning, akin to the way that documents are utterances over a language of words. Inspired by
advancements originating in the natural language processing community, we ask whether it is possible and practical to learn contextual embeddings
of events akin to WordToVec, and to construct aggregate representations of entire traces and ultimately collections of similar traces.

We make k contributions:
