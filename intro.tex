%-------------------------------------------------------------------------------
\section{Introduction}
%-------------------------------------------------------------------------------
Here are three problems: 
\begin{enumerate}
\item What went wrong? For some failed execution, given its trace, can we hint at the possible cause of failure by comparing shapes of graphs from failed and successful interactions?
\item How does the system behave at scale? How can we artificially generate new traces to test scale out?
\item What just happened? Trace classification i.e. Given a set of traces and their corresponding classes, can we predict the class of a new trace? 
\end{enumerate}

For all of the above problems, the solutions are a combination of:
\begin{itemize}
\item Tracing: Implementing tracing in systems means that we can now get detailed information of the system execution at the request level, which can give us a better understanding of the system.
\item Graph representations: How can we generate graph representations which are useful in comparing and reasoning about graphs drawn from traces?
\end{itemize}

Why representational learning?: \newline
Typically, to obtain a vector representation of a graph, G=(V,E), we might extract features relevant to the problem we are trying to solve, such as maximum depth, average in-degree, average fan-out, etc. Features extracted are dependent both on the family of graphs under consideration and also upon the specific task. Therefore, this method is tedious and may need to repeated for each graph family and/or task.

Representational learning enables us to learn representations that capture structural context and/or corresponding attribute values automatically and are general enough to be used in a variety of downstream tasks. \newline

Contributions:
\begin{itemize}
\item Representational learning far outperforms a naive approach \kam{Work in progress. Claim as yet unproven }
\item Provide initial results of the use of graph representations for a variety of use cases as hinting at possible causes of failures, classification and clustering.
\end{itemize}
