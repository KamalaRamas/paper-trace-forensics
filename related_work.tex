%-------------------------------------------------------------------------------
\section{Related Work}
%-------------------------------------------------------------------------------

\ash{
Post factum analysis of traces is important because they contain useful information like causality of events, resource consumption, errors, performance data and labeled metdata about components involved in the execution.
In large systems the size of the traces makes it impractical for a developer to do any manual analysis. This motivates the need for automated analysis of large corpus of traces to derive useful insights.
There are many goals for analysis for traces, including finding performance bottlenecks, system recovery \& repair, debugging, resource usage, service degradation analysis.
Many of these use cases use some form of classification or clustering. We descrobe some of the prior work in this area and highlight the drawbacks and explain how our approach overcomes some of the hurdles.
}